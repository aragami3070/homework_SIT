\texttt{$\gamma$}  "--- отношение теплоемкостей

\texttt{$m$} "--- масса газа

начальные состояния:

\begin{enumerate}
    \item {$V_0$ "--- объем}
    \item {$p_0$ + $H$ "--- давление, где ($p_0$ "--- атмосферное давление, $H$ "--- разность уровней жидкости в манометре) }
    \item {$T_0$ "--- температура}
\end{enumerate}

состояние в конце процесса:

\begin{enumerate}
    \item {$V_1$ "---  объем}
    \item {$p_0$ + $H'$ "--- давление, где ($p_0$ "--- атмосферное давление, $H'$ "--- разность уровней жидкости в манометре) }
    \item {$T_1$ "--- температура}
\end{enumerate}

\begin{equation}
    \label{eq : second_eq}
    p_0 \cdot V_0^\gamma = (p_0 + H)V_0^\gamma
\end{equation}
\begin{equation}
    \label{eq : third_eq}
    (p_0 + H) \cdot V_0 = (p_0 + H')V_1
\end{equation}

из \ref{eq : second_eq} и \ref{eq : third_eq}

\begin{equation*}
    (\frac{V_0}{V_1})^\gamma = \frac{p_0}{p_0 + H} 
\end{equation*}

\begin{equation*}
    (\frac{V_0}{V_1})^\gamma = \frac{(p_0 + H')^\gamma }{(p_0 + H)^\gamma}
\end{equation*}

откуда 

\begin{equation*}
    \frac{p_0}{p_0 + H}  = \frac{(p_0 + H')^\gamma }{(p_0 + H)^\gamma}
\end{equation*}

Логарифмируя это выражение, получим:

\begin{equation*}
    ln \frac{p_0}{p_0 + H}  = \gamma ln \frac{(p_0 + H')}{(p_0 + H)}
\end{equation*}

Определим отсюда:

\begin{equation*}
    \gamma = \frac{ln \frac{p_0}{p_0 + H}}{ln \frac{(p_0 + H')}{(p_0 + H)}} = \frac{ln (1 - \frac{H}{p_0 + H})}{ln(1- \frac{(H - H')}{(p_0 + H)})}
\end{equation*}

разлагая логарифмы в ряды по формуле:

\begin{equation*}
    ln (1-x)  = -x + \frac{1}{2} x^2 -\frac{1}{3} x^3 +....
\end{equation*}

и ограничиваясь только первыми членами разложения, найдем приближенно:

\begin{equation*}
    \gamma = \frac{- \frac{H}{p_0 + H}}{- \frac{(H - H')}{(p_0 + H)}} = \frac{H}{H-H'}
\end{equation*}
